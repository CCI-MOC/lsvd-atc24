\section{Introduction}

- Distributed networked storage has high demands on bandwidth and on the storage medium

- Traditionally solved with a write-back cache, but that makes state management hard

- Small writes are incredibly costly, you need consistency/replication semantics on every request

- We propose a log-structured virtual disk on a central storage gateway, on top
of immutable objects

- Follow-up to LSVD\cite{lsvd}

Benefits:

- We get statistical multiplexing between different clients

- Built on top of immutable objects = trivial caching

- Batched writes for good write performance

- Snapshots and clones are trivial

- Elastic backend to prevent low-space GC thrashing

\orran{
  I think we need to take head on that this is an extension of the previous work, and not hide that...

  I would start off by motivating problem, i.e., we want disaggregated storage for a wide range of machines, BW, latency....

  Recent work developed a log structured approach; point to our eurosys paper, but the morons that did that work ended up with a system with some serious limitations.

  We adopted the implementation from that paper, and moved it to a gateway. Then say the work we did, i.e., shared cache, log replication, integrate into a SPDK, moved to a pure user-level implementation to support that, ...

  Then say, this new system, Gateway Log-Structured Virtual Device (GLSVD) has a number of major advantages over previous system:

  - enables machiens to boot from this
  - allows NVME to be shared across multiple physical machines;
  - sharing of cache
  - compared system to an SSD backend...
}

We find....

