\section{Introduction}

Remote block storage devices are the foundation of many cloud environments as
a simple, widely-compatible source of persistent storage. As such, any
implementation must contend with a wide range of workloads and use cases,
providing similar semantics and performance with traditional local storage.

Existing work \cite{lsvd} demonstrated the viability of a log-structured virtual
disk (LSVD) on a single machine by developing a kernel module that exposes a
virtual block device. The module batches writes to the backend on a local write
log that is periodically flushed to the backend.  This approach requires
modifications to the client and is thus not widely compatible with existing
technologies.

We adopt the implementation in the existing paper and move it to a central
gateway instead. We retain all the benefits of the existing implemenation, such
as dramatically improved write performance and efficiency, while also enabling
multiple new opportunities such as the ability to share the cache between
different clients.

We thus propose a \textbf{Gateway Log-Structured Virtual Device (GLSVD)}
\isaac{we really need a better name}. We show that our approach brings with it
the following benefits:

\isaac{TODO: figure out a better way to word and organise this}

\begin{enumerate}

  \item \textbf{Statistical multiplexing} between clients

  \item \textbf{Shared cache:} The gateway can cache data from multiple clients
        and thus amortise the cost of a read across multiple clients. This is
        especially useful for workloads with high locality, such as booting a cluster
        of identical virtual machines, where many of the blocks are shared and thus
        take up no additional cache space per client.

  \item \textbf{Elastic backend:} Unlike traditional log-structured systems, the
        size of our backend is not fixed, and may grow and shrink as according to the
        needs the system. This allows us to prevent low-space garbage collection
        thrashing, which is a common problem in other log-structured systems.

  \item \textbf{Trivial snapshots and clones:}


\end{enumerate}

Our prototype demonstrates read performance on par with that of a local SSD for
workloads that fit within our local cache while maintaining the benefit of
improved write performance and efficiency.

\isaac{briefly summarise setup and comparison}

\isaac{briefly summarise rest of paper}

\orran{
  I think we need to take head on that this is an extension of the previous
  work, and not hide that...

  I would start off by motivating problem, i.e., we want disaggregated storage
  for a wide range of machines, BW, latency....

  Recent work developed a log structured approach; point to our eurosys paper,
  but the morons that did that work ended up with a system with some serious
  limitations.

  We adopted the implementation from that paper, and moved it to a gateway. Then
  say the work we did, i.e., shared cache, log replication, integrate into a
  SPDK, moved to a pure user-level implementation to support that, ...

  Then say, this new system, Gateway Log-Structured Virtual Device (GLSVD) has a
  number of major advantages over previous system:

  - enables machiens to boot from this
  - allows NVME to be shared across multiple physical machines;
  - sharing of cache
  - compared system to an SSD backend...
}

We find....

